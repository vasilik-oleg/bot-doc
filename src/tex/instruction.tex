\csection{Соглашение об ответственности}

<<Финансовая Компания <<Викинг>> не несет ответственности за умышленные попытки пользователя дестабилизировать работу торгового робота.

\csection{Архитектура}

Торговый робот \emph{<<\AppName>>} состоит из двух частей: серверной и клиентской. Серверная часть робота -- это то место, где непосредственно выполняется сам торговый
алгоритм.  Клиентская часть робота  -- это графический интерфейс пользователя (GUI). Отделение графического интерфейса от торгового алгоритма позволяет разнести по разным
физическим устройствам торговую часть и интерфейс пользователя, так торговый алгоритм может быть запущен на физическом сервере или виртуальной машине под управлением ОС
семейства Linux в зоне колокации Московской биржи, в то время как графический интерфейс пользователя устанавливается на персональный компьютер клиента.

Принцип совместной работы двух частей торгового робота (серверной и клиентской) состоит в следующем: все настройки всегда хранятся на клиентской части, если вы запускаете
серверную часть то она запускается не инициализированной настройками (т.е. торговых портфелей в ней нет и торговать она не будет). При первом подключении клиентской части
робота к серверной, серверная часть получает настройки торговых портфелей и хранит их до своего (серверной части) выключения.

\emph{Следует четко понимать следующие моменты, связанные с работой робота:}
\vspace{-\topsep}
\begin{itemize}
\setlength{\parskip}{0pt}
\setlength{\itemsep}{0pt plus 1pt}
\item \emph{если вы отключаете клиентскую часть от серверной, но не выключаете при этом серверную часть, то все настройки в ней остаются и если торговля не остановлена, то она
	будет продолжаться при выключенной клиентской части;}
\item \emph{если вы подключаетесь клиентской частью к серверной, которая не перезапускалась и уже заинициализирована настройками, то настройки загрузятся уже из серверной
	части в клиентскую, именно поэтому все настройки портфелей клиентской части рекомендуется проводить будучи подключенным к серверной части робота.}
\end{itemize}
\vspace{-\topsep}

Далее речь пойдет только о настройке клиентской части, при этом подразумевается, что серверная часть запущена, и подключение ко всем требуемым торговым площадкам установлено.\newline

\noindent\textit{\underline{Важно:} ЗАПРЕЩЕНО два и более одновременных подключения клиентской части к серверной, одновременное подключение приведет к потере данных из-за
конфликта графических частей при подключении к серверной.}

\csection{Настройка подключения}

Для начала работы с торговым роботом необходимо запустить клиентскую часть. Открывшийся графический интерфейс будет иметь вид, как показано на рисунке~\ref{fig:gui}.

\begin{figure}[h]
\centering
%\includegraphics[width=5cm]{pict1}
\includegraphics[width=0.9\textwidth]{pict1}
%\subfloat[Триангуляция сечения]{\includegraphics[width=5cm]{pic3}}
\caption{Графический интерфейс пользователя до подключения}
\label{fig:gui}
\end{figure}

Если это первый запуск GUI, то необходимо настроить подключение графической части к роботу, для этого нужно выбрать следующий \raisebox{-1mm}{\includegraphics[height=6mm]{manageconnects}}
пункт главного меню \raisebox{-1mm}{\includegraphics[height=6mm]{main}}, в открывшемся окне указать требуемые
данные (Рис.~\ref{fig:connection}). Обычно это \emph{Host} и \emph{Robot port}. Если вы подключаетесь с использованием SSL (например, вам выдали файл, содержащий все неодходимые ключи и SSL сертификаты, т.е. файл с расширением \textit{.xml})
то поставьте галку \textit{Use SSL} и укажите путь к полученному файлу в поле \textit{Certs path}.
Далее следует нажать кнопку \emph{Apply} для применения изменений.
\begin{figure}[h!]
\centering
%\includegraphics[width=5cm]{pict1}
\includegraphics[width=0.3\textwidth]{pict4}
%\subfloat[Триангуляция сечения]{\includegraphics[width=5cm]{pic3}}
\caption{Настройка подключения к серверной части робота}
\label{fig:connection}
\end{figure}
При последующих запусках робота настройку соединения выполнять не требуется.
\begin{figure}[h!]
\centering
%\includegraphics[width=5cm]{pict1}
\includegraphics[width=0.9\textwidth]{pict3}
%\subfloat[Триангуляция сечения]{\includegraphics[width=5cm]{pic3}}
\caption{Графический интерфейс пользователя, соединение установлено}
\label{fig:gui_connected}
\end{figure}

\begin{figure}[h!]
\centering
%\includegraphics[width=5cm]{pict1}
\includegraphics[width=0.9\textwidth]{pict7}
%\subfloat[Триангуляция сечения]{\includegraphics[width=5cm]{pic3}}
\caption{Графический интерфейс пользователя, попытка установки соединения}
\label{fig:gui_not_connected}
\end{figure}
Для подключения к серверной части робота необходимо нажать на кнопку подключения \raisebox{-1mm}{\includegraphics[height=6mm]{connect}}, после чего GUI примет вид как представлено на
рисунке~\ref{fig:gui_connected}, если вместо этого GUI имеет вид как представлено на рисунке~\ref{fig:gui_not_connected}, то следует проверить указанные настройки подключения
(Рис.~\ref{fig:connection}), если настройки верны, необходимо обратиться к контактному лицу. Если подключение выполнено успешно, то следует выполнить обновление списка
финансовых инструментов, нажав кнопку \raisebox{-1mm}{\includegraphics[height=6mm]{refresh}}.\newline

\noindent\textit{\underline{Важно:} Изменение всех торговых настроек следует проводить ТОЛЬКО при наличии подключения к серверной части робота.}

\csection{Настройка портфелей и торговые функции}

После того, как выполнено подключение к серверной части робота, и робот имеет вид как показано на рисунке~\ref{fig:gui_connected}, можно приступать к настройке нового
торгового портфеля. Если это не первый запуск робота, то в таблице будут отображены уже существующие портфели. Для этого необходимо нажать кнопку добавления портфеля
\raisebox{-1mm}{\includegraphics[height=6mm]{add}}, в открывшемся окне нужно указать необходимые настройки (Рис.~\ref{fig:add_portfolio}). В каждом торговом портфеле может быть несколько
инструментов, один из которых является главным -- по нему рассчитывается позиция всего портфеля. Кроме создания нового портфеля доступны функции редактирования существующего
портфеля, создания его копии, а так же удаления портфеля, для этого служат кнопки \raisebox{-1mm}{\includegraphics[height=6mm]{options}}, \raisebox{-1mm}{\includegraphics[height=6mm]{cprecord}} и
\raisebox{-1mm}{\includegraphics[height=6mm]{del}} соответственно. Чтобы отредактировать портфель, создать его копию или удалить его, необходимо чтобы нужный портфель был выбран в
таблице. Так же существует возможность экспортировать настройки портфелей (\raisebox{-1mm}{\includegraphics[height=6mm]{export}} в главном меню \raisebox{-1mm}{\includegraphics[height=6mm]{main}}) и импортировать настройки портфелей из имеющегося файла
настроек (\raisebox{-1mm}{\includegraphics[height=6mm]{import}} в главном меню \raisebox{-1mm}{\includegraphics[height=6mm]{main}}). При импорте будут добавлены портфели с уникальными именами, имеющиеся портфели изменены или удалены НЕ будут.

Когда у бумаги до даты экспирации остается 3 и менее дней, если у этого инструмента есть дата экспирации, робот оповестит вас об этом перечеркнув имя бумаги.

Для просмотра сделок, совершенных роботом, предусмотрена кнопка \raisebox{-1mm}{\includegraphics[height=6mm]{money}}, в открывшейся таблице отображаются раздвижки по совершенным сделкам
(если строк в таблице становится более $100000$ ''старые'' строки будут автоматически удаляться).

При закрытии GUI все настройки сохраняются, и при следующем включении будут доступны все созданные портфели. После добавления портфелей основной экран робота будет выглядеть
как показано на рисунке~\ref{fig:gui_portfolio}. Изменение всех торговых настроек следует проводить только при наличии подключения к серверной части робота.

\begin{figure}[h!]
\centering
%\includegraphics[width=5cm]{pict1}
\includegraphics[width=0.9\textwidth]{pict5}
%\subfloat[Триангуляция сечения]{\includegraphics[width=5cm]{pic3}}
\caption{Добавление нового портфеля}
\label{fig:add_portfolio}
\end{figure}
\begin{figure}[h!]
\centering
%\includegraphics[width=5cm]{pict1}
\includegraphics[width=0.9\textwidth]{pict9}
%\subfloat[Триангуляция сечения]{\includegraphics[width=5cm]{pic3}}
\caption{Графический интерфейс пользователя}
\label{fig:gui_portfolio}
\end{figure}

Для включения и остановки торговли по всем портфелям одновременно служат кнопки \raisebox{-1mm}{\includegraphics[height=6mm]{start}} и \raisebox{-1mm}{\includegraphics[height=6mm]{stop}}
соответственно. Если необходимо задать расписание работы робота, т.е. отрезки времени, когда будет производиться торговля, то нажатием на кнопку
\raisebox{-1mm}{\includegraphics[height=6mm]{clock}} нужно вызвать окно настройки расписания. В расписании следует указывать время для того же часового пояса, который выбран на машине,
где запущена серверная часть торгового робота (при торговле на Московской бирже, предполагается использование московского времени), в не зависимости от того, какое время
задано на клиентской машине. Кнопка \raisebox{-1mm}{\includegraphics[height=6mm]{stop1}} выключает торговлю всех портфелей, причем выключает и расписание.
Для настройки уведомлений по портфелям при резком изменении значений некоторых параметров используйте кнопку \raisebox{-1mm}{\includegraphics[height=6mm]{message}}.

В строке состояния в нижней части главного окна программы расположен суммарный счетчик транзакций (то есть попыток выставления и снятия заявок) по всем транзакционным
подключениям к бирже (\textit{Transactions}). Также в строке состояния расположен индикатор состояния подключения всех подключений торгового робота к бирже (\textit{Gate}),
при наведении курсора мыши на индикатор появится всплывающая подсказка с подробной расшифровкой состояния всех подключений.\newline

\noindent\textit{\underline{Важно:} По умолчанию вы можете добавить не более $50$-ти портфелей и при этом не более $100$-а финансовых инструментов суммарно во все портфели,
при превышении указанных чисел, превысивший ограничения портфель будет автоматически отключен.}

\csection{Редактор формул}

\ifdefined \Ramiz
\else
\noindent\textit{\underline{Важно}: функционал, описываемый в данном разделе, доступен НЕ во всех версиях робота.\newline}
\fi

В окне натроек портфеля расположена кнопка \textit{Edit formulas}, которая открывает редактор формул данного портфеля (Рис.~\ref{fig:edit_formulas}). В левом
столбце в редакторе расположены только те имена параметров портфеля, которые являются формулами и используются, исходя из текущих настроек портфеля (те параметры,
которые сейчас используются не как формулы НЕ будут отображены). Чтобы открыть
редактор для параметра из левого столбца, надо сделать двойной клик левой кнопки мыши на имене соответствующего параметра, новый редактор откроется ниже всех,
уже открытых редкторов. Используя пункты меню: \textit{Securities}, \textit{Portfolios}, \textit{Deals}, -- вы можете добавлять параметры финансовых инструментов,
портфелей и сделок, не набирая соответствующий код (подходит для не опытных пользователей). Формулы пишутся на языке программирования C++, вы пишете только
тело функции и должны вернуть значение типа \textit{double}. Более подробно API, доступное в редакторе формул, будет описано в отдельном разделе.

\begin{figure}[h!]
\centering
\includegraphics[width=0.9\textwidth]{pict10}
\caption{Редактор формул портфеля}
\label{fig:edit_formulas}
\end{figure}

\csection{Сброс статусов заявок робота}

В торговом роботе есть ''механизм отслеживания заявок'', бывают ситуации, например, когда происходят сбои в работе биржи, что заявка выставляется роботом, а потом снимается
биржей, никак не информируя робот о снятии. В таком случае заявка ''зависает'' в своем текущем, внутреннем для робота, статусе. Для сброса статусов всех заявок портфеля есть
кнопка \raisebox{-1mm}{\includegraphics[height=6mm]{terminate}}. Пользоваться это кнопкой можно ТОЛЬКО в КРАЙНИХ случаях, когда торговля по портфелю отключена и Вы уверены что нет ни одной
активной заявки по данному портфелю, в противном случае робот потеряет активные заявки, что приведет к неправильной позиции по бумагам в роботе.

\csection{Неторговые функции}

Во время работы робота могут возникать различные ошибки при выставлении и удалении заявок, такие ситуации не являются нештатной работой робота и могут быть вызваны причинами,
не связанными с некорректной работой торгового робота, например, отсутствием денег на счете клиента. Для настройки уведомлений об ошибках, информационных сообщениях и поведения
главного окна программы выберите следующий \raisebox{-1mm}{\includegraphics[height=6mm]{configure}} пункт главного меню \raisebox{-1mm}{\includegraphics[height=6mm]{main}}.
Все ошибки и еще некоторую информацию, связанную с работой программы, можно посмотреть в логе,
нажав на кнопку \raisebox{-1mm}{\includegraphics[height=6mm]{log}} (если строк в таблице становится более $100000$ ''старые'' строки будут автоматически удаляться).
Некоторые слишком часто приходящие сообщения будут отображаться в логе не все, а $1$ раз в $10$ секунд и будут отмечены в конце сообщения символом \textit{xN}, где \textit{N} -- количество непоказанных сообщений.
Время в логе -- это время прихода сообщения из серверной части в клиентскую, это локальное время машины, на
которой запущена клиентская часть. По этому времени НЕЛЬЗЯ судить о скорости работы робота.

Стоит заметить, что все таблицы робота настраиваемые. Можно менять местами столбцы, а так же отключать ненужные. Для перетаскивания столбца достаточно зажать левую
кнопку мыши на его заголовке и перетащить столбец в нужное место таблицы. Для отключения лишних столбцов нужно сделать правый клик мышью на заголовке таблицы, после этого
откроется контекстное меню, как показано на рисунке~\ref{fig:check}. В этом же контекстном меню есть пункты применения фильтра к таблице и копирования/экспорта таблицы в $csv$ формате.
Для таблицы сделок, совершенных роботом, и таблицы лога в контекстном меню есть возможность очистить таблицу.
\begin{figure}[h!]
\centering
%\includegraphics[width=5cm]{pict1}
\includegraphics[width=0.4\textwidth]{pict8}
%\subfloat[Триангуляция сечения]{\includegraphics[width=5cm]{pic3}}
\caption{Отключение столбцов таблицы}
\label{fig:check}
\end{figure}

%В таблице сделок, совершенных роботом, и в таблице логов отображаются данные, получаемые $online$ с серверной части, поэтому если закрыть клиентскую часть и открыть ее снова
%то эти таблицы будут пустыми, и данные начнут поступать только при совершении новых сделок и добавлении новых записей в лог. Данные в этих таблицах не сохраняются на локальной
%машине, в виду того, что этих данных может быть очень много и чтобы не засорять оперативную память компьютера. В описываемых таблицах отображается только время сделки или
%добавления записи в лог (без даты), но при сортировке по времени дата также учитывается поэтому не обязательно очищать таблицы при наступлении нового дня, так как данные будут
%отображаться в правильном порядке (с учетом не только времени, но и даты).

\phantomsection
\label{keys}\csection{Настройка ключей подключений}

Для того, чтобы добавить в робот ключи, сгенерированные при помощи программы \href{http://fkviking.ru/downloads/index.php?get=CryptoKeyGen}{CryptoKeyGen}, нужно выбрать следующий \raisebox{-1mm}{\includegraphics[height=6mm]{key}}
пункт главного меню \raisebox{-1mm}{\includegraphics[height=6mm]{main}}, в открывшемся окне поставить флаг \textit{Use keys}, чтобы включить использование ключей, и
добавить ключи из соответствующих файлов (\textit{*\_key\_for\_trader.key}) используя кнопку \textit{Add key}. Содержимое столбца таблицы \textit{Comment} -- произвольный пользовательский
комментарий к ключу, содержимое столбца \textit{Key} -- непосредственно сам ключ.

\phantomsection
\label{manageconns}\csection{Добавление/удаление подключений к биржам криптовалют}

Для того, чтобы добавить/удалить в робот подключение к криптовалютной бирже, нужно выбрать следующий \raisebox{-1mm}{\includegraphics[height=6mm]{manageconns}}
пункт главного меню \raisebox{-1mm}{\includegraphics[height=6mm]{main}}. В верхней половине открывшегося окна расположен список \textit{market data} подключений, которые можно
включить или выключить, используя флаг \textit{Disabled}, в нижней половине открывшегося окна расположен список транзакционных подключений, кнопки для удаления/добавления
транзакционных подключений и кнопка обновления списка текущих подключений. При добавлении нового транзакционного подключения необходимо выбрать тип подключения (биржу)
и заполнить список необходимых параметров, при наведении на инонку \raisebox{-1mm}{\includegraphics[height=6mm]{help}} отображается необходимая информация о каждом из параметров.
Двойной клик по ячейке в столбце статуса соединения позволяет переподключить данное подключение.\newline

\noindent\textit{\underline{Важно:} Нельзя удалить транзакционное подключение на котором есть активные заявки; подключения, которые запрещены на серверной части,
не будут добавлены; по умолчанию нельзя добавить больше трех транзакционных подключений. Ключи подключений будут автоматически добавлены в менеджер ключей,
описанный в \hyperref[keys]{\ref{keys}предыдущем} разделе}

\phantomsection
\label{positions}\csection{Отображение позиции на биржах криптовалют}

Для того чтобы посмотреть текущие позиции на бирже (поддерживаются не все биржи, список поддерживаемых бирж обновляется) используйте кнопку \raisebox{-1mm}{\includegraphics[height=6mm]{poses}}. В открывшемся окне вы можете выбрать
интересующее подключение и посмотреть позиции по бумагам и/или баланс по деньгам в соответствующих таблицах (отображаемые значения зависят от конкретной биржи).\newline

\noindent\textit{\underline{Важно:} Настоятельно не рекомендуется все время держать данное окно открытым, т.к. это довольно сильно увеличивает количество передаваемой информации от серверной части робота к графической (особенно если у
вас много открытых позиций по выбранному подключению), все равно позиции по всем подключениям одновременно вы не увидите}

\csection{Обновление программы и документация}

Клиентская часть робота автоматически проверяет наличие обновлений при запуске и далее $1$ раз в час (убедитесь, что адрес \href{http://fkviking.ru/downloads}{http://fkviking.ru/downloads}
НЕ добавлен в черный список вашего фаервола). При наличии обновлений на панели появится иконка \raisebox{-1mm}{\includegraphics[height=6mm]{achievements}} (вместо иконки меню помощи
\raisebox{-1mm}{\includegraphics[height=6mm]{help-contents}}), нажав на которую вы сможете открыть меню помощи и, выбрав соответствующий пункт меню, скачать инсталлятор новой версии программы.
Для проверки наличия обновлений программы в ручном режиме необходимо выбрать следующий \raisebox{-1mm}{\includegraphics[height=6mm]{download}} пункт
меню помощи.

Для того чтобы открыть файл с документацией, используйте следующий \raisebox{-1mm}{\includegraphics[height=6mm]{help-contents}} пункт меню помощи, документация откроется в просмотрщике $pdf$ файлов,
установленном в системе по-умолчанию.\newline

\noindent\textit{\underline{Важно:} При наличии обновлений клиентской части рекомендуется ВСЕГДА обновлять программу, во избежании конфликта версий и невозможности подключения
''старой'' клиентской части к более новой (обновляемой без Вашего участия) серверной части. Серверная часть робота обновляется независимо от клиентской и обычно ее обновляют
ночью, после окончания торговых сессий на всех рынках, поэтому рекомендуется НЕ оставлять активных заявок после окончания торгов, так как если серверная часть будет
перезапущена, то она уже не ''увидит'' оставленных заявок, и эти заявки будут потеряны роботом и не будут корректно обработаны, что приведет к несоответствию позиции в роботе
и реальной позиции на счете (если эти заявки исполнятся).}

\phantomsection
\label{round_trip}
\csection{Подсчет скорости транзакций}

При наведении курсора мыши на надпись, отображающую число транзакций, совершенных роботом (\textit{Transactions}), появится всплывающая подсказка, в которой отображена
статистика скорости транзакций для каждого транзакционного подключения. Данные отображены в таблице, строки таблицы соответствуют имеющимся в роботе транзакционным
подключениям, столбцы таблицы:
\begin{enumerate}
	\item \textit{connect} -- название подключения;
	\item \textit{all} -- суммарное количество транзакций по подключению;
	\item \textit{> 0.5} -- количество транзакций с round-trip-ом больше $0.5$ мс по подключению;
	\item \textit{> 1} -- количество транзакций с round-trip-ом больше $1$ мс по подключению;
	\item \textit{> 2} -- количество транзакций с round-trip-ом больше $2$ мс по подключению;
	\item \textit{> 4} -- количество транзакций с round-trip-ом больше $4$ мс по подключению;
	\item \textit{> 8} -- количество транзакций с round-trip-ом больше $8$ мс по подключению;
	\item \textit{> 16} -- количество транзакций с round-trip-ом больше $16$ мс по подключению;
	\item \textit{adds} -- суммарное количество успешных транзакций на выставление заявки по подключению;
	\item \textit{dels} -- суммарное количество успешных транзакций на удаление заявки по подключению;
	\item \textit{moves} -- суммарное количество успешных транзакций на перемещение заявки по подключению;
	\item \textit{rejects} -- суммарное количество отвергнутых транзакций по подключению;
	\item \textit{avg} -- среднее время round-trip-а транзакций (в микросекундах) по подключению;
	\item \textit{avga} -- среднее время round-trip-а транзакций на выставление заявки (в микросекундах) по подключению;
	\item \textit{avgd} -- среднее время round-trip-а транзакций на удаление заявки (в микросекундах) по подключению;
	\item \textit{avgm} -- среднее время round-trip-а транзакций на перемещение заявки (в микросекундах) по подключению;
	\item \textit{avgr} -- среднее время round-trip-а отвергнутых транзакций (в микросекундах) по подключению.
\end{enumerate}

\noindent\textit{\underline{Важно:} Время замеряется именно для round-trip-а, т.е. это время между моментом отправки транзакции и моментом прихода ответного
сообщения на данную транзакцию.}

\csection{Настройка telegram-бота}

Telegram-бот доступен по адресу \href{https://telegram.me/FKVikingBot}{https://telegram.me/FKVikingBot}.
Для настройки telegram-бота необходимо подключиться клиентской частью робота к серверной, далее выбрать следющий \raisebox{-1mm}{\includegraphics[height=6mm]{telegram}}
пункт главного меню \raisebox{-1mm}{\includegraphics[height=6mm]{main}}
и запросить ключ вашего робота, нажав в открывшемся окне на кнопку \textit{''Generate new key''}, после этого скопировать полученный ключ (он выделен жирным шрифтом) в чат
telegram-бота в диалог добавления нового робота (ключ действителен $3$ минуты, если не успели, еще раз нажмите на кнопку \textit{''Generate new key''}). Если вы все сделали
правильно, то в списке роботов в telegram-боте появится добавленный вами робот.\newline

\noindent\textit{\underline{Важно:} Если вы случайно или специально перегенерируете ключ робота, робот останется ''привязан'' к telegram-боту. ''Отвязать''
робота от telegram-бота можно только выбрав пункт ''Remove robot'' в меню бота.}

\phantomsection
\label{move}\csection{Использование приказа ''переместить заявку''}

На некоторых подключениях реализована отправка приказа ''переместить заявку'' (иногда эта команда также называется изменением заявки). Использование
данного приказа позволяет сократить количество транзакций и увеличить отношение количества сделок к количеству транзакций, особенно при котировании,
а так же увеличить время нахождения заявок в рынке. Так как особенности использования данной команды отличаются на разных рынках и типах подключений,
то, в зависимости от подключения, данный приказ может применяться только для первой ноги портфеля или же для инструментов обеих ног. На данный момент
отправка такой команды реализована для FIX-подключений фондового и валютного рынков Московской биржи, а так же plaza2 и TWIME подключений срочного
рынка Московской биржи. Для FIX-протокола срочного рынка Московской биржи использование данной команды не поддерживается.

Использование приказа ''переместить заявку'' для поддерживаемых подключений не является обязательным. Эта возможность включается и отключается на
серверной стороне. Если необходимо включить или отключить эту функциональность, следует обратиться в поддержку. В графической части никаких изменений
при этом не требуется.

\phantomsection
\label{orderbook}\csection{Особенности использования торговых стаканов инструментов}

В большинстве подключений к торговым площадкам стаканы по всем бумагам или полный лог заявок (далее всегда будем называть это просто стаканом в независимости от
реально используемого потока) приходят в одном потоке, т.е. нельзя получать данные по конкретным бумагам, получать вы всегда будете все данные, а использовать можете
только для нужных бумаг. Робот НЕ строит стакан сразу по всем бумагам, строит только по используемым в портфелях бумагам. Для добавления новой бумаги в список бумаг,
по которым строится стакан, необходимо переоткрыть поток с текущим слепком стакана, а потом накатывать на него обновления из потока инкрементальных обновлений. В
момент получения потока со слепком стакана стакан в роботе временно перестает обновляться по всем используемым бумагам, он начнет обновляться только после того как
будет закрыт поток со слепком стакана (это может занять некоторое время, зависящее от общего количества бумаг, приходящих в данном потоке, и размера стаканов) и начнут
применяться инкрементальные обновления. Соответственно, в момент переоткрытия стакана вы можете наблюдать отсутствие цен по бумагам.

Стакан в роботе всегда переоткрывается в следующих случаях:
\begin{enumerate}
	\item создание нового портфеля;
	\item добавление бумаги в портфель;
	\item снятие флага \textit{Disabled} с портфеля;
	\item пропуски в номерах сообщений инкрементального потока обновлений стакана в UDP подключениях к биржам (чем больше у вас портфелей и бумаг, тем выше
		вероятность возникновения этих пропусков).
\end{enumerate}

\phantomsection
\label{problems}\csection{Возможные проблемы и их решение}

Описание наиболее распространенных ошибок в работе робота и способов их устранения.
\begin{itemize}
\item \textbf{Проблема}

При подключении к серверной части меняются настройки торговых портфелей.

\textbf{Решение}

Все правильно, так и должно быть. Все изменения настроек портфелей рекомендуется производить только когда выполнено подключение к серверной части. В противном случае после подключения
загрузятся те настройки, которые были переданы на сервер ранее.
\item \textbf{Проблема}

Торговля включается и выключается самопроизвольно.

\textbf{Решение}

Проверьте, не задано ли расписание торговли.
\item  \textbf{Проблема}

Отсутствует нужный инструмент в списке инструментов.

\textbf{Решение}

Список бумаг в роботе обновляется каждое утро в 9:25 по Москве, чтобы выгрузить список бумаг из робота в графическую часть
необходимо нажать кнопку \raisebox{-1mm}{\includegraphics[height=6mm]{refresh}}.

\noindent\textit{\underline{Важно:} время на серверах с крипто роботами $-3$ часа от Москвы. Если вы не видите какой-то бумаги в списке бумаг (при этом вы обновили список),
а эта бумага уже есть на бирже, то либо дождитесь указанного выше времени и бумага добавится сама, либо переподключите дата подключение (это можно сделать из менеджера подключений,
доступно НЕ для всех подключений) и после этого обновите список бумаг в графической части робота.}

%\item  \textbf{Проблема}
%
%Робот перестал выставлять заявки по первой бумаге при включенном режиме $Wait \; hedge$ и позиция не захеджирована.
%
%\textbf{Решение}
%
%В такой ситуации надо выровнять позицию через торговый терминал и задать нужную (уже ровную) позицию в роботе. После этого робот продолжит торговать.

\item  \textbf{Проблема}

Робот ставит заявки по второй ноге в ''непонятном'', неправильном объеме.

\textbf{Решение}

Данная ситуация может возникать после какого-либо сбоя на бирже, когда биржа не прислала необходимую информацию по заявкам робота и в роботе ''зависли'' внутренние статусы
заявок. В данной ситуации необходимо остановить торговлю по проблемному портфелю, убедиться что по данному портфелю нет активных заявок в рынке и сбросить статусы заявок,
нажав кнопку \raisebox{-1mm}{\includegraphics[height=6mm]{terminate}}.

\end{itemize}

\phantomsection
\label{errors}\csection{Наиболее распространенные ошибки, возникающие при работе программы, и способы их устранения}

\textit{\underline{Важно:} большинство возникающих ошибок в работе робота, это не ошибки самого робота, а проблемы со связью, между клиентской и серверной частями, вызванные
плохим качеством интернет соединения, и ошибки выставления заявок, например, из-за нехватки денег на счете.}\newline

\noindentТорговые ошибки:

\begin{itemize}
\item \textbf{Ошибка}

\textit{Order adding error on *, error: REASON\_NO\_MONEY}

или

\textit{order adding error on *, order exceeds limit}

\textbf{Описание и способ устранения}

Нехватка денег на счете. Проверьте хватает ли у вас денег на счете для выставления заявки по данной бумаге данного объема. Торговля будет остановлена автоматически.

\textit{\underline{Важно:} С биржи приходит много разных сообщений об ошибке выставления заявки, в роботе они сгруппированы и под ошибку REASON\_NO\_MONEY попадает
несколько разных сообщений от биржи, например: ''Account has insufficient Available Balance'' и ''Value of position and orders exceeds position Risk Limit''. Эти сообщения
присылает биржа в ответ на выставление заявки, робот сам никогда не возвращает ошибку REASON\_NO\_MONEY, это только сообщение с биржи. Почему приходит данное
сообщение спрашивайте у соответствующей биржи. Сообщение, непосредственно пришедшее с биржи, вы можете найти в логе графической части робота.}

\item \textbf{Ошибка}

\textit{order adding error on *, continue trading, error: REASON\_FLOOD}

\textbf{Описание и способ устранения}

Флуд-контроль, превышен лимит максимального разрешенного количества транзакций (удалений/снятий заявок) в секунду для данного логина или биржа прислала сообщение
в духе ''Биржа перегружена, попробуйте позже'' или некоторые другие сообщения от биржи, которые говорят о том, что биржа сейчас испытывает некоторые ''трудности'',
но в принципе работает. Для данного робота $30$ транзакций в
секунду (единица транзакционной активности Московской биржи, принятая на текущий момент) -- это элементарно.

\item \textbf{Ошибка}

\textit{order adding error on *, continue trading, error: REASON\_UNDEFINED}

\textbf{Описание и способ устранения}

Вы получите эту ошибку в том случае, если ошибку выставления заявки, которую присылает биржа, робот видит впервые (в документации бирж не все сообщения описывают почему-то).
В случае данной ошибки в лог графической части приходит сообщение, которое прислала биржа, читайте его внимательнее, абсолютно не факт что что-то не так с роботом и нужно писать
в поддержку, и у бирж бывают проблемы. Если же вы понимаете что данное сообщение должно быть как-то обработано роботом, например, как REASON\_FLOOD или REASON\_NO\_MONEY,
то пишите в поддержку.

\item \textbf{Ошибка}

\textit{order adding error on *, continue trading, error: REASON\_PRICE\_OUT\_OF\_LIMIT}

\textbf{Описание и способ устранения}

Данная ошибка означает, что цена заявки вне лимита, и это нормально, так может быть, т.к. цены бумаг в некоторых случаях рассчитываются и зависят от ваших настроек.

\item \textbf{Ошибка}

\textit{order adding error on *, continue trading, error: REASON\_CROSS}

\textbf{Описание и способ устранения}

Данная ошибка возникает когда вы пытаетесь торговать сами с собой. Когда происходит ошибка робот первую ногу не трогает, а вторую старается выставить на один шаг в сторону улучшения цены.

\end{itemize}

\noindentНеторговые ошибки:

\begin{itemize}
\item \textbf{Ошибка}

\textit{Error on sendMsg (<class 'ConnectionRefusedError'>, ConnectionRefusedError(10061, 'Подключение не установлено, т.к. конечный компьютер отверг запрос на подключение',
None, 10061, None))}

или 

\textit{Error on sendMsg (<class 'ConnectionRefusedError'>, ConnectionRefusedError(111, 'Connection refused'))}

\textbf{Описание и способ устранения}

Нет соединения с роботом, скорее всего робот просто не запущен в данный момент. Проверьте запущена ли серверная часть робота, если у вас нет на это прав, то обратитесь к
администратору за информацией о состоянии робота. Если данная ошибка продолжалась порядка нескольких минут и после этого клиентская часть снова подключилась к серверной, то,
возможно, серверная часть по какой-то причине была перезапущена.

\item \textbf{Ошибка}

\textit{Error on sendMsg (<class 'socket.timeout'>, timeout('timed out',))}

или 

\textit{Error on sendMsg (<class 'OSError'>, OSError(113, 'No route to host'))}

или

\textit{Error on sendMsg (<class 'OSError'>, OSError(101, 'Network is unreachable'))}

\textbf{Описание и способ устранения}

Нет соединения с компьютером на котором запущена серверная часть. Вероятнее всего указан не верный IP-адрес серверной части в настройках подключения, либо на удаленной машине
запущен фаервол, который запрещает входящие подключения, либо отсутствует соединение с интернетом (или с локальной сетью) на машине где запущена клиентская часть, либо
отсутствует соединение с интернетом (или с локальной сетью) на машине где запущена серверная часть. Если отсутствует соединения с сетью на машине, на которой запущена
серверная часть, то сообщение всегда \textit{socket.timeout}.

\item \textbf{Ошибка}

\textit{Cann't parse message from server  (<class 'xml.etree.ElementTree.ParseError'>, ParseError('no element found: line 1, column 0',))}

\textbf{Описание и способ устранения}

Данная ошибка сопутствует предыдущей ошибке, так как программа не получила валидное сообщение от серверной части и не смогла его корректно обработать.

\item \textbf{Ошибка}

\textit{Multiple client connection is not allowed!!!}

\textbf{Описание и способ устранения}

Запрещено одновременное подключение нескольких клиентских частей к серверной. Убедитесь что у Вас нигде не запущена вторая клиентская часть уже подключенная к серверной части.
Подождите $10$ секунд, и попробуйте подключиться снова.

\item \textbf{Ошибка}

\textit{Error on sendMsg (<class 'OSError'>, OSError(9, 'Bad file descriptor'))}

или

\textit{Error on sendMsg (<class 'OSError'>, OSError(10038, 'Сделана попытка выполнить операцию на объекте, не являющемся сокетом', None, 10038, None))}

\textbf{Описание и способ устранения}

Попытка подключения используя уже закрытый сокет. Скорее всего сокет был принудительно закрыт серверной частью из-за попытки подключиться одновременно второй клиентской частью

\item \textbf{Ошибка}

\textit{Error reading application version file "*.version" from site, during update check: (<class 'urllib. error.URLError'>, URLError( TimeoutError(10060, 'Попытка
установить соединение была безуспешной, т.к. от другого компьютера за требуемое время не получен нужный отклик, или было разорвано уже установленное соединение из-за неверного отклика уже
подключенного компьютера', None, 10060, None)))}

или

\textit{Error reading application version file "*.version" from site, during update check: (<class 'urllib. error.URLError'>, URLError( gaierror(-5, 'No address associated with hostname')))}

\textbf{Описание и способ устранения}

Программа не смогла проверить обновление на сервере из-за проблем со связью.

\end{itemize}
%\section*{Введение}
%Основная идея метода Давыдова (метода крупных частиц) состоит в расщеплении по физическим процессам исходной нестационарной системы уравнений Эйлера,

%Расчет каждого временного шага (вычислительного цикла) разбивается на три этапа:
%\begin{enumerate}
%\item Эйлеров этап, на котором пренебрегается всеми эффектами, связанными с перемещением элементарной ячейки (потока массы через границы ячеек нет), и
%учитываются эффекты ускорения жидкости лишь за счет давления. Здесь для крупной частицы определяются промежуточные значения искомых параметров потока.
%\item Лагранжев этап, на котором при движении жидкости вычисляются потоки массы через границы эйлеровых ячеек.
%\item Заключительный этап, на котором определяются в новый момент времени окончательные значения газодинамических параметров потока на основе законов
%сохранения массы, импульса и энергии для каждой ячейки и всей системы в целом на фиксированной расчетной сетке.
%\end{enumerate}

%\section*{Начальные и граничные условия}
%В случае однородного потока газа в начальный момент времени во все ячейки расчетной области помещаются некоторые начальные значения для  газодинамических
